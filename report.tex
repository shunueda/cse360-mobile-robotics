%%%%%%%%%%%%%%%%%%%%%%%%%%%%%%%%%%%%%%%%%%%%%%%%%%%%%%%%%%%%%%%%%%%%%%%%%%%%%%%%
%2345678901234567890123456789012345678901234567890123456789012345678901234567890
%        1         2         3         4         5         6         7         8

\documentclass[letterpaper, 10 pt, conference]{ieeeconf}  % Comment this line out
% if you need a4paper
%\documentclass[a4paper, 10pt, conference]{ieeeconf}      % Use this line for a4
% paper

\IEEEoverridecommandlockouts                              % This command is only
% needed if you want to
% use the \thanks command
\overrideIEEEmargins
% See the \addtolength command later in the file to balance the column lengths
% on the last page of the document

\usepackage[utf8]{inputenc}
\usepackage[T1]{fontenc}

% The following packages can be found on http:\\www.ctan.org
%\usepackage{graphics} % for pdf, bitmapped graphics files
%\usepackage{epsfig} % for postscript graphics files
%\usepackage{mathptmx} % assumes new font selection scheme installed
%\usepackage{mathptmx} % assumes new font selection scheme installed
\usepackage{amsmath}
%\usepackage{amssymb}  % assumes amsmath package installed
\usepackage{listings}
\usepackage{color}
\definecolor{dkgreen}{rgb}{0,0.6,0}
\definecolor{gray}{rgb}{0.5,0.5,0.5}
\definecolor{mauve}{rgb}{0.58,0,0.82}
\lstset{frame=tb,
    language=Python,
    aboveskip=3mm,
    belowskip=3mm,
    showstringspaces=false,
    columns=flexible,
    basicstyle={\small\ttfamily},
    numbers=none,
    numberstyle=\tiny\color{gray},
    keywordstyle=\color{blue},
    commentstyle=\color{dkgreen},
    stringstyle=\color{mauve},
    breaklines=true,
    breakatwhitespace=true,
    tabsize=3
}

%%%%%%%%%%%%%%%%

\title{\LARGE \bf Title}

\author{Shun Ueda}

\begin{document}
    \maketitle
    \thispagestyle{empty}
    \pagestyle{empty}


    \section{Excercise 1}
    For excerise 1 and 3, following functions are implemented.

    \begin{lstlisting}[label={lst:lstlisting}]
    def sense(x):
        return x

    def simulate(Δt, x, dx):
        x += Δt * dx
        return x
    \end{lstlisting}

    Control function is implemented as follows.

    \begin{lstlisting}[label={lst:lstlisting1}]
    import numpy as np

    def control(t, y):
        ux = -4 * sin(t)
        uy = 2 * cos(t)
        return np.array([ux, uy])
    \end{lstlisting}

    Since the major and minor axis are set to 4m and 2m respectively, we know that the functoin takes the form of $ux = 4 \sin(t)$ and $uy = 2 \cos(t)$.
    Moreover, since the path is counterclockwise, we adjust the function to $ux = -4 \sin(t)$ and $uy = 2 \cos(t)$.


    %    \addtolength{\textheight}{-12cm}
%    \begin{thebibliography}{99}
%        \bibitem{c1} G. O. Young, ``Synthetic structure of industrial plastics (Book style with paper title and editor),'' in Plastics, 2nd ed. vol. 3, J. Peters, Ed. New York: McGraw-Hill, 1964, pp. 15--64.
%    \end{thebibliography}
\end{document}
